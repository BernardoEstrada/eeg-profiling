%%%%%%%%%%%%%%%%%%%%%%%%%%%%%%%%%%%%%%%%%%%%%%%%%%%%%%%%%%%%%%%%%%%%%%%%%%%%%%%
%2345678901234567890123456789012345678901234567890123456789012345678901234567890
%        1         2         3         4         5         6         7         8

\documentclass[letterpaper, 10 pt, conference]{ieeeconf}  % Comment this line out
                                                          % if you need a4paper
%\documentclass[a4paper, 10pt, conference]{ieeeconf}      % Use this line for a4
                                                          % paper
\usepackage[utf8]{inputenc}  % Ng Edit for accents in spanish
\usepackage[spanish]{babel}  % Ng Edit for accents in spanish
\usepackage{hyperref}        % Ng Edit for adding urls
\usepackage{graphicx}        % Ng Edit for adding graphics
\IEEEoverridecommandlockouts                              % This command is only
                                                          % needed if you want to
                                                          % use the \thanks command
\overrideIEEEmargins
% See the \addtolength command later in the file to balance the column lengths
% on the last page of the document

\def\equationautorefname~#1\null{(#1)\null}%to use parenthesis in eqs.


% The following packages can be found on http:\\www.ctan.org
%\usepackage{graphics} % for pdf, bitmapped graphics files
%\usepackage{epsfig} % for postscript graphics files
%\usepackage{mathptmx} % assumes new font selection scheme installed
%\usepackage{times} % assumes new font selection scheme installed
%\usepackage{amsmath} % assumes amsmath package installed
%\usepackage{amssymb}  % assumes amsmath package installed

\title{\LARGE \bf
Proyecto Integrador para el Desarrollo de Soluciones Empresariales  Políticas generales del curso 
}

%\author{ \parbox{3 in}{\centering Huibert Kwakernaak*
%         \thanks{*Use the $\backslash$thanks command to put information here}\\
%         Faculty of Electrical Engineering, Mathematics and Computer Science\\
%         University of Twente\\
%         7500 AE Enschede, The Netherlands\\
%         {\tt\small h.kwakernaak@autsubmit.com}}
%         \hspace*{ 0.5 in}
%         \parbox{3 in}{ \centering Pradeep Misra**
%         \thanks{**The footnote marks may be inserted manually}\\
%        Department of Electrical Engineering \\
%         Wright State University\\
%         Dayton, OH 45435, USA\\
%         {\tt\small pmisra@cs.wright.edu}}
%}
%Benjamín Valdés, Pedro Pérez, Arturo Escobedo , Alfonso Gómez, András Takacs, Silvana de Guile Rocío Aldeco.
\author{Benjamín Valdés$^{1}$,  José Antonio Cantoral$^{1}$, \& Agustín Dominguez$^{1}$ % <-this % stops a space
\thanks{$^{1}$Departamento de  Computación y Mecatrónica,  Tecnológico de Monterrey Campus Querétaro, Epigmenio Gonzalez \#500 76130 Querétaro, México.
\newline Asesores: Arturo Pérez, Alfonso Gómez, Pedro Pérez.
\newline Clave: TC3007 - TC3054 
\newline Unidades: 12 unidades
\newline Oficina: Edificio 2, 3er piso.
\newline {\tt\small bvaldesa at itesm.mx}}%
}


\begin{document}



\maketitle
\thispagestyle{empty}
\pagestyle{empty}


%%%%%%%%%%%%%%%%%%%%%%%%%%%%%%%%%%%%%%%%%%%%%%%%%%%%%%%%%%%%%%%%%%%%%%%%%%%%%%%%
\begin{abstract}

Este documento describe el curso de Proyecto Integrador. El documento incluye las actividades que llevaremos a cabo durante la materia, las fechas, la forma de trabajo, las restricciones del curso y los criterios de evaluación. Así mismo el documento sirve como machote (template) del formato que se espera usen en sus entregables. El código fuente de las políticas se encuentra disponible en línea en caso de que quieran usar \textit{Latex} para falicitar dar formato a sus documentos, en caso de que decidan no utilizar \textit{Latex}, se espera que cumplan con formatos de revista o journal en sus entregables.  
\end{abstract}


%%%%%%%%%%%%%%%%%%%%%%%%%%%%%%%%%%%%%%%%%%%%%%%%%%%%%%%%%%%%%%%%%%%%%%%%%%%%%%%%
\section{¡ Bienvenido al curso de Proyecto Integrador!}

Para acreditar este curso es necesario desarrollar un proyecto que solucione un problema, que sea probado, analizado, y comparado con otras soluciones. Todo el análisis que llevarás acabo deberá ir plasmado en una tesina que irás redactando durante el semestre. Una tesina es un documento donde se describe una pequeña contribución a un área de conocimiento. Para apoyarte en esta tarea, trabajarás con un asesor que te supervisará, aconsejará, y evaluará durante el semestre. Al terminar este curso, deberás ser capaz de crear una solución computacional para un problema no estructurado, relevante y real. Durante el curso desarrollarás capacidades de pensamiento crítico, pensamiento científico, integración de conocimiento, redacción avanzada, y comunicación oral. Aquí pondrás en práctica de forma integral gran parte de lo que has aprendido en la carrera. Es un curso demandante, que requiere compromiso derivado del trabajo de investigación y desarrollo, por ello se te sugiere mantener una comunicación constante y clara con tu asesor, que será tu principal guía, cuya opinión tiene un peso fuerte sobre la calificación final de la materia.\\

El formato de este documento está basado en artículos de congreso de las IEEE. Tiene pequeñas modificaciones para el español que pueden verse en los comentarios del código fuente (dicen Ng Edit). La intención es que te sirva este documento como ejemplo y referencia para tu propuesta inicial de investigación que se entrega en 2 semanas. En caso de no querer usar latex puedes descargar de \url{https://www.ieee.org/conferences/publishing/templates.html}  un template equivalente en Word, el formato del entregable también se califica. 

Este documento no sólo tiene ejemplo de formato sino también de elementos comunes en publicaciones científicas, por ejemplo los autores se colocan (usualmente) como primer autor, el que escribe el documento y tiene la mayor contribución en el trabajo, y como último autor el investigador de mayor rango involucrado en el trabajo (jefe del departamento o laboratorio involucrado en el trabajo). También  encontrarás ejemplos de tablas, listas, e imágenes. El código fuente para este documento lo podrás encontrar en \url{https://www.overleaf.com/read/wygkvkvmhkxz}. Cualquier duda que tengas sobre \textit{latex} o el formato procura resolverla en las primeras sesiones.  


\section{Metodología de trabajo}

\subsection{primeras sesiones}
Las primeras sesiones asistirás a clase para capacitarte de forma intensa en metodología de investigación y escritura de una tesina. Estas sesiones están diseñadas para equiparte con lo que vas a necesitar para realizar tu tesina. Gran parte del trabajo posterior a estas sesiones será auto dirigido, así que aprovecha estas 3 semanas.\\
\subsection{trabajo autodirigido/asesores}
Una vez entregada tu propuesta inicial, trabajarás de forma individual con tu asesor y solo será necesario que te presentes en los tracks que se llevan a cabo al inicio y al final del semestre.\\ 

\subsection{presentaciones en tracks}
Para poder tener derecho a cada presentación se requiere el Visto Bueno (VoBo) del asesor (El asesor debe de haber visto y autorizado lo que el alumno va a presentar).\\
\begin{itemize}
  \item Presentación en track de propuestas. (10\% final)
  \item Presentación en track de resultados.  (15\% final)
\end{itemize}
 \\
Los tracks serán dirigidos por un profesor y están enfocados al \textit{peer-review} (se evaluarán entre ustedes). Al inicio de cada Track el profesor de la sesión (chair) explica cuanto tiempo tiene cada persona para presentar y para preguntas. Normalmente el tiempo será de 10 minutos de presentación y 5 minutos de preguntas. El profesor en cada Track designará un responsable del tiempo, y está encargado de que no se rebase el tiempo para cada segmento. Encaso de pasarse de tiempo se les interrumpe la presentación para pasar a las preguntas. Es importante tomar en cuenta los siguiente aspectos de los tracks:\\
\begin{itemize}
    \item \textbf{Tiempo} La duración de cada presentación debe apegarse a la duración establecida, y una vez concluido el tiempo, el chair podrá interrumpir la presentación en caso de que ésta no haya concluido. El tiempo asignado a cada estudiante ya contempla el tiempo de instalación y preparación. Terminado el tiempo de preguntas ya no se pueden hacer más preguntas o comentarios, en caso de haber más preguntas o dudas se les pide a los estudiantes esperar a que terminen todas las demás presentaciones.
    \item \textbf{Evaluación Peer-Review} Durante el tiempo de preguntas (5 min.) tendrán que calificar a la persona que está presentando usando un formaulario como este: \url{https://forms.gle/pCry2McPo4o3M4Wk7}. El link real se les proveerá al inicio de la presentación. Su participación en el track se mide mediante la retroalimentación honesta que proporcionen en el formato. Como se espera que participen en todo el track, por cada presentación que no retroalimenten perderán un porcetanje de la calificación de su presentación, si se detecta que sus comentarios y calificaciones son deshonestas, superficiales o incongruentes, se tomarán como nulas y también perderán el porcentaje correspondiente. La intención de este ejercicio es brindarles la oportunidad de participar en una actividad de revisión entre pares como sería en el caso de la revisión de un artículo de un congreso.
    \item \textbf{Retroalimentación} Después del día de las presentaciones, el coordinador hará llegar los comentarios y calificación a cada asesor con el cuál podrán platicar ustedes y ver cómo mejorar su trabajo.
\end{itemize}

\section{Evaluación del Curso}

Tu evaluación se divide en 3 partes, la primera es la propuesta que entregas al coordinador (15\% final), la segunda  son las presentaciones (25\% final), y la tercera es tu investigación que evalúa tu asesor (60\% final). 

\subsection{Restricciones para pasar el Curso}

\textbf{
\begin{itemize}
	\item Para poder tener derecho a presentar, en cada presentación se requiere el VoBo del asesor (El asesor debe de haber visto y autorizado lo que el alumno va a presentar). En caso de no presentar, se perderá la ponderación de la presentación correspondiente (0 puntos).
    \item Para tener derecho a que se califique la tesina se necesita tener un prototipo o desarrollo que respalde lo escrito. Si no hay prototipo o desarrollo, se perderán 55 de los 60 puntos que califica su asesor. Es decir, su máxima calificación posible en el curso será de 45 sobre 100.
    \item Las tesinas inconclusas (sin evaluación o resultados) se calificarán sobre 30 puntos en lugar de sobre 60. Es decir, su máxima calificación posible en el curso será de 70 sobre 100.
    \item En caso de utilizar parte del trabajo o su totalidad en otra materia, se deberá notificar y contar con el VoBo de su asesor y del otro profesor para que sea calificado. De no ser así serán sujetos a Faltas de Integridad Académica.
    \item Comportamientos que vayan en contra de la visión de la institución o de las competencias que se desean  desarrollar, serán reportados al coordinador de la materia. El coordinador convocará a un comité de profesores para analizar el caso y determinar una sanción adecuada y será sujeto a ser tratado como una Faltas de Integridad Académica. Ejemplos de comportamientos que caen dentro de este rubro incluyen, pero no se limitan a: abusos verbales, faltas de respeto, deshonestidad, incongruencia en lo que se dice y hace, manipulación, entre otras.
    \item Este curso es por proyecto, por lo que no se puede acreditar con examen de última materia y no hay extensiones de tiempo.
\end{itemize}
}

\subsection{Aspectos que se califican en tu trabajo}
\begin {itemize}
	\item Su capacidad de análisis.
    \item Su capacidad de comunicar sus ideas por escrito.
    \item Su capacidad de presentar su proyecto de forma clara, concreta y concisa.
    \item El diseño de herramientas auxiliares (Diagramas, gráficas, póster)
    \item Sus capacidades técnicas.
    \item Dificultad.
    \item Originalidad.
    \item La calidad de su trabajo.
\end {itemize}

\subsection{ponderaciones}

En la tabla \ref{t1}  se muestran las ponderaciones para el curso. Los parciales son indicadores de desempeño, pero no se usan para calificar la calificación final del curso. Lo que cuenta para el final del curso es la segunda columna de la tabla.   

\begin{table}[h]
  \caption{Ponderacón de Actividades}
  \label{t1}
  \begin{center}
    \begin{tabular}{|c||c|}
      \hline
        Evaluación parcial & Evaluación Final \\ 
        (la boleta es un aproximado) & (calificación real del curso) \\
      \hline
        Avance asesor  100\%  &  \(\alpha\) Propuesta 15\% \\
        & \(\beta\) Presentación track propuesta 10\% \\
        & \(\gamma\) Presentación track final 15\% \\
        & \(\delta\) Actividades con Asesor  60\% \\
        & \(\epsilon\) Extras: Apoyo al Departamento 5\% \\
       \hline
    \end{tabular}
  \end{center}
\end{table}

\begin{equation}
\label{eq:eq1}
\alpha * 0.15 + \beta * 0.10 + \gamma * 0.15 + \delta * 0.60 + \epsilon =  1.05    
\end{equation} 


Donde en \autoref{eq:eq1} se puede ver que cada elemento suma al 100\% de la materia, y cumplir con alguna actividad de apoyo al departamento otorga 5 puntos extras sobre la calificación final. Las actividades de apoyo cambian cada semestre, pueden ser participar en los proyectos designados para la materia, presentar en expo día, generar contenido para novus o material de calidad académica que pueda ser publicado como un artículo de revista arbitrada.

\section{Entregables}

Para que tu trabajo sea recibido y calificado tiene que cumplir con los siguientes requisitos:
\begin{itemize}
     % \item La tesina engargolada, impresión por ambos lados. Aquí hay esta formato: 
     \item La tesina o reporte escrito en la carpeta de la materia en formato \url{https://www.overleaf.com/read/thbgjhkfwnwh}. En algunos casos para los trabajos de mayor calidad y que cuenten con la autorización del asesor será posible entregar un artículo para publicación en lugar de la tesina. El artículo debe de estar en el formato del journal al que será sometido. También debe de contener toda la información solicitada en la tesina a manera de anexos. Por ejemplo, en un artículo corto de journal de 10 cuartillas, se esperarán anexos del estado del arte a detalle, de metodología de evaluación, detalles de solución, ecuaciones y experimentos que aprox. tomarían 20 - 30 cuartillas.
     \item Se debe de entregar un repositorio público con los códigos fuentes (como Github) con el trabajo realizado.
     \item Se debe entregar evidencia del correcto funcionamiento del prototipo. Para esto se sugiere usar un video demostrando el correcto funcionamiento o si el servicio está disponible en linea el link al servidor correspondiente.
\end{itemize}

\section{Fechas importantes}
\begin{itemize}
	\item	\textbf{Semana 1 introducción al curso.} 
        \item	\textbf{Semana 2 selección de tema.} Al final de la segunda semana enviarán sus propuesta de investigación. Los asesores consultarán las propuestas y tendrán 2 días para seleccionar al estudiante con el cual deseen trabajar.
        \item	\textbf{Semana 3 asesores.} En la semana 3 se les notificará a los alumnos quién será su asesor.
        \item	\textbf{Semana 4 tracks de propuestas.} Presentación en tracks de la propuesta de investigación. (30 de agosto)
        \item	\textbf{Semana 14 sesión opcional.} Dudas sobre cómo grabar demos y recomendaciones para la presentación final. 
        \item	\textbf{Semana 15 tracks finales.} (15 de noviembre)
	   \item	\textbf{Semana 16 entrega final.} Tesina, repositorio, evidencias y apoyo departamento. (22 de noviembre). \textbf{Después de este día no hay revisiones}, aprovechen la retroalimentación continua de su asesor.
\end{itemize}

\section{Sesiones presenciales}
\subsection{sesión 1}
	\begin{itemize}
		\item Introducción al curso.  Lectura de políticas y formas de evaluar.
		\item Explicación sobre que es una tesina. \emph{¿De dónde vienen? ¿Para qué sirven? ¿Qué competencias desarrollarán en ellas? ¿Qué se busca en una tesina? ¿Cómo se trabajan? ¿Cómo están estructuradas?}
		\item Analizar ejemplos de tesinas anteriores. Ver la estructura que siguen.  \emph{¿Qué? ¿Quién? ¿Por qué? ¿Cómo? ¿Qué paso? ¿Qué significa? ¿Qué cosa nueva sabemos y qué sigue?}
		\item Nota Una página de puro texto times new roman tamaño 12 contiene 400 palabras en promedio. El tamaño de un capítulo no es lo más importante, pero es un indicador sobre la profundidad del trabajo y la calidad del análisis.
		\item \textbf{Introducción}. De que se va a tratar esta tesina. (1000 a 1500 palabras).
		\item \textbf{Marco Teórico}. Que hay en el área, analizar y comparar, no sólo citar (2500 a 5000 palabras). \emph{Conceptos necesarios, actualidad de la industria, trabajos relevantes en el área, comparación de trabajos, análisis sobre los mejores, ¿Según quién son los mejores? ¿Por qué son los mejores? ¿Cuál es la relación entre ellos? ¿Qué cosas no resuelven o no contemplan? \textbf{¿Cómo se evalúan?}}
		\item \textbf{Problema}. \emph{Porqué lo que hay no es suficiente o como se puede mejorar, (por que vale la pena lo que voy a hacer) (1000 a 2000 palabras).}
		\item \textbf{Solución}. \emph{Qué hice o cómo lo hice. Y porque creo que va a funcionar (Lo que proponga tienen que evaluarlo).  (2500 a 5000 palabras).}
		\item \textbf{Diseño de evaluación o metodología de experimento}. \emph{Cómo voy a probar que lo que digo es cierto. (Data sets, casos de prueba, pruebas de usuario, etc.).  Qué pruebas han hecho los demás. (1000 a 2000 palabras).}
		\item \textbf{Resultados}. \emph{¿Qué salió en los experimentos o pruebas que hice?  (1000 a 2000 palabras)}
		\item \textbf{Interpretación}. \emph{¿Qué significan los resultados, comparación con otros trabajos? ¿Por qué funcionó o no funciono? ¿Qué estamos aprendiendo de esto, o qué descubrimos? (1000 a 2000 palabras).}
		\item \textbf{Conclusión}. \emph{¿Qué fue lo que aprendimos de este trabajo y qué se puede hacer con lo aprendido? ¿Qué sigue? (1000 a 2000 palabras).}
		\item \textbf{Propuesta}. Mostrar ejemplo de qué es una propuesta. Describir las secciones y objetivos.
		\item Editor sugerido para publicar propuesta, tesina y su formato \url{https://www.overleaf.com} Desde aquí pueden consultar y copiar este documento \url{https://www.overleaf.com/read/wygkvkvmhkxz}
\end{itemize}

\subsection*{Tarea} 

\begin{itemize}
		\item leer una de las tesinas o artículos de ejemplo (revisar los capítulos, ver estructura, análisis y técnicas de escritura).
		\item Buscar y leer un artículo científico de revista/journal del área o tema que te interesa, solamente en revistas oficiales.
            \item Tutorial de acceso a W.O.S. con la biblioteca digital, antes de la clase 3:30.
\end{itemize}

\subsection{sesión 2}
\begin{itemize}
		\item Dudas sobre la estructura de las tesinas leídas.
		\item Preguntas sobre sesión anterior 
		\item La controversia con respecto a las fuentes: \cite{news} 
        \item Tipos de fuentes:
        \begin{itemize} 
        		\item  (n00b) Wikipedia
                \item  (n00b) google
           		\item  (n00b) blogs
                \item  (n00b ish) revistas de divulgación  \url{https://ccc.inaoep.mx/~ksapiens/ks/31/ks31-compacta.pdf}
				\item  (n00b ish) revistas no indexadas \url{http://www.arxiv-sanity.com/}
				\item  (n00b ish) libros texto \cite{book1}
            	\item  (n00b ish) periódico \cite{news}
                \item  (1337) Manuales Técnicos \cite{Manual-Esp1}
                \item  (1337) Estándares oficiales \cite{Manual-Esp1}
                \item  (1337) revistas/journals Indexadas Q1 Q2 Q3 \cite{journal1}
                \item  (n00b ish) conferencias/proceedings. \cite{conference1}
        \end{itemize}
		\item ¿Para qué sirve cada una?
		\item ¿Qué es válido citar y qué no?
		\item scimago journal ranking \url{https://www.scimagojr.com/}
		\item web of science \url{https://vimeo.com/500664640}
		\item Manejador de Referencias \url{https://www.mendeley.com/}
		\item Como leer un artículo/paper de forma crítica. Abstract, Conclusion, después el resto.
		\item Preguntas clave
		\item Lectura de un artículo/paper con formato de análisis \url{https://cutt.ly/yjRN8CA}. 
		\item Ver los temas de los profesores.
\end{itemize}

\subsection*{Tarea} 
\begin{itemize}
	\item ver los videos o leer las propuestas de los profesores.
	\item Buscar y leer un artículo científico de revista/journal del área o tema que te interesa.
\end{itemize}

\subsection{sesión 3}
\begin{itemize}
		\item Dudas sobre las líneas de proyectos.
		\item Preguntas sesión anterior 
		\item Selección de línea de proyecto (explicación de diferentes líneas).
        \begin{itemize}
        	\item \textbf{pregunta de investigación}, hipótesis. -> creo que A ocurre por B
            \item \textbf{problemática}, objetivo de investigación -> A debe de resolver problema B
	    \end{itemize}
		\item Selección de tema.
		\item Escribir propuestas 20 min.
		\item Analizar propuestas con la siguiente lista de cotejo \url{https://cutt.ly/YjRZPwA}. 10 min.
\end{itemize}

\subsection*{Tarea} 
\begin{itemize}
	\item Buscar y leer un artículo científico de revista/journal del área o tema que te interesa.
	\item Comenzar a escribir propuesta
    \item La propuesta se evaluará con la siguiente lista de cotejo \url{https://cutt.ly/YjRZPwA}.
\end{itemize}

\subsection{Sesión 4}
\begin{itemize}
		\item Dudas sobre fuentes y lectura de artículos/papers.
		\item Preguntas sesión anterior
		\item Trabajo en sus propuestas iniciales 15 \%, las propuestas deben de estar en el formato de este documento. Crea una copia en \textit{Overleaf} y edítalo a tu conveniencia o descarga un template en word de  \url{https://www.ieee.org/conferences/publishing/templates.html}.
		\item Definir alcances y acotar los trabajos. (Explicación sobre las funciones de los asesores).
		\item Ese día hasta la media noche se entrega su propuesta inicial en la carpeta de drive que lleva su matrícula en: \url{https://drive.google.com/drive/folders/1JIC8Uea8hmF6UStiFnoxL3ObOaaO3A7K?usp=sharing}. Entre mejor esté escrito y más claro sea, será más atractivo para los asesores trabajar con ustedes y es el 15\% de su calificación final.
\end{itemize}

\subsection*{Tarea} 
\begin{itemize}
	\item Buscar y leer un artículo científico de revista/journal del área o tema que te interesa.
	\item Terminar de escribir propuesta (15\% final)
    \item La propuesta se evaluará con la siguiente lista de cotejo \url{https://cutt.ly/YjRZPwA}.
\end{itemize}

\subsection{Sesión 5}
\begin{itemize}
		\item Explicar dinámica de los tracks y evaluación de los tracks, \url{https://forms.gle/pCry2McPo4o3M4Wk7}.
		\item Resolver dudas adicionales que los estudiantes pudiesen tener.
		\item Notificación de asesor.
\end{itemize}

\addtolength{\textheight}{-12cm}   % This command serves to balance the column lengths
                                  % on the last page of the document manually. It shortens
                                  % the textheight of the last page by a suitable amount.
                                  % This command does not take effect until the next page
                                  % so it should come on the page before the last. Make
                                  % sure that you do not shorten the textheight too much.

%%%%%%%%%%%%%%%%%%%%%%%%%%%%%%%%%%%%%%%%%%%%%%%%%%%%%%%%%%%%%%%%%%%%%%%%%%%%%%%%



%%%%%%%%%%%%%%%%%%%%%%%%%%%%%%%%%%%%%%%%%%%%%%%%%%%%%%%%%%%%%%%%%%%%%%%%%%%%%%%%




\section*{Agradecimientos}

Los autores quieren agradecer a todos los asesores del curso en los últimos años Pedro Pérez, Arturo Escobedo , Alfonso Gómez, Rocío Aldeco, Eduardo Juarez, Jesus Arturo Perez, András Takacs y Silvana de Guile, por su continuo apoyo en la materia y en mejorar la experiencia de los estudiantes.



%%%%%%%%%%%%%%%%%%%%%%%%%%%%%%%%%%%%%%%%%%%%%%%%%%%%%%%%%%%%%%%%%%%%%%%%%%%%%%%%

\begin{thebibliography}{99}

\bibitem{news} J. Travis,	In survey, most give thumbs-up to pirated papers, Sience, May 6  American Association for the Advancement of Science, 2016. available at \url{http://www.sciencemag.org/news/2016/05/survey-most-give-thumbs-pirated-papers}
\bibitem{journal1} R. Vilalta and Y. Drissi, “A perspective view and survey of meta-learning,” Journal of Artificial Intelligence in Education, vol. 18, no. 2, pp. 77–95, 2002.
\bibitem{conference1} S. Sahebi, Y. Huang, and P. Brusilovsky, “Predicting Student Performance in Solving Parameterized Exercises,” in Proceedings of the 12th International Conference on Intelligent Tutoring Systems, 2014, pp. 496–503.
\bibitem{book1} J. Piaget and B. Inhelder, The Growth of Logical Thinking from Childhood to Adolescence. New York: Basic Books, 1958.
\bibitem{Manual-Esp1} “IMS ePortfolio Information Model. Version 1.0 Final Specification.,” vol. 2009, no. November 18. IMS Global Learning Consortium, 2005.

\end{thebibliography}
%%%%%%%%%%%%%%%%%%%%%%%%%%%%%%%%%%%%%%%%%%%%%%%%%%%%%%%%%%%%%%%%%%%%%%%%%%%%%%%%

\pagebreak %%sugest a page break
\section*{APPENDIX:Políticas del Departamento de Computación y Mecatrónica}

   
  \begin{figure}[thpb]
      \centering
      \includegraphics[scale=0.2]{logo_tec}
      \caption{Este es el logo del tec que deben utilizar en caso de hacer posters}
      \label{figuraLogoTec}
   \end{figure}
 
\textbf{Exámenes.}
Los exámenes podrán ser presentados solamente en la fecha estipulada. El no presentar un examen implica una calificación de NP (No Presentó).\\
	El cambio de fecha de algún examen parcial deberá realizarse, a petición de los estudiantes, durante las dos primeras semanas de clase. Éste se hará sólo si se cuenta con el consenso del grupo y del profesor.\\
\textbf{Tareas, Actividades y Proyectos.}\\
	Toda tarea, actividad y/o proyecto tendrá su fecha y horario de entrega que es inamovible. Vencido este término no se recibirán más entregas.\\
	Todas las tareas son individuales a menos que explícitamente se pida trabajar en grupo.\\
\textbf{Redacción y Organización.}\\
	La mala redacción, organización y ortografía en la elaboración de tareas, proyectos, presentaciones y exámenes, será causa de penalización en la calificación correspondiente.\\
\textbf{Calificaciones.}\\
	Las calificaciones parciales y final se expresan en escala de uno a cien.\\
	La calificación mínima aprobatoria es 70 (SETENTA).\\
\textbf{Faltas a la Integridad Académica en Tareas, Proyectos o Exámenes.}\\
	Las faltas a la integridad académica, como la copia o tentativa de copia en cualquier tipo de examen o actividad de aprendizaje; el plagio parcial o total; facilitar alguna actividad o material para que sea copiada y/o presentada como propia; la suplantación de identidad; falsear información; alterar documentos académicos; vender o comprar exámenes o distribuirlos mediante cualquier modalidad; hurtar información o intentar sobornar a un profesor o cualquier colaborador de la institución; entre otras acciones más son consideradas faltas grave. Cuando un alumno cometa un acto contra la integridad académica, se le asignará una calificación reprobatoria a la actividad, examen, período parcial o final. La calificación reprobatoria asignada por el profesor será inapelable, y a esta sanción se sumarán las otras posibles que determine el Comité de Integridad Académica de Campus. Esto tal como lo indica el Reglamento Académico en su CAPÍTULO IX Faltas a la integridad académica \\
\textbf{Baja de Materias.}\\
	La fecha límite para darse de baja de cualquier materia es un día antes de terminar clases.\\

\end{document}
